% This file makes use of sydetex.

% sydetex is free software: you can redistribute it and/or modify
% it under the terms of the GNU General Public License as published by
% the Free Software Foundation, either version 3 of the License, or
% any later version.

% sydetex is distributed in the hope that it will be useful,
% but WITHOUT ANY WARRANTY; without even the implied warranty of
% MERCHANTABILITY or FITNESS FOR A PARTICULAR PURPOSE.  See the
% GNU General Public License for more details.

% You should have received a copy of the GNU General Public License
% along with sydetex.  If not, see <https://www.gnu.org/licenses/>.


\documentclass[12pt]{article}
\usepackage[utf8]{inputenc}
\usepackage{setspace}
\usepackage{CJK}



% THIS IS WHERE ALL THE SYDE STUFF IS
\usepackage{sydestyle}


% Adding numbering to subsubsections
\setcounter{secnumdepth}{3}
\setcounter{tocdepth}{3}

% Reference tooling package settings - point to where the files are
\graphicspath{{figures/}}

\addbibresource{references.bib}

\makeglossaries
\input{glossary.tex}


% hyperref is a prissy princess, so we import it here to prevent weirdness with sydetex.sty
\usepackage{hyperref}
\hypersetup{
    bookmarks=false
}



% PERSONAL DETAILS HERE, this sets up title page
\def\reportTitle{Designing a Circuit Protection Sensor Module}
\def\reportDueDate{September 6, 2018}
\def\studentName{Si Teng Liu}
\def\studentNumber{20473965}
\def\company{FireBright Corporation}
\def\lastSemester{3A}



% Form fitting ;)
\title{\reportTitle}
\author{\studentName}
\studentnumber{\studentNumber}
\lastsemester{\lastSemester}
\date{\reportDueDate}
\employer{\company}
\employeraddress{
\begin{CJK}{UTF8}{gbsn}
松江区 莘砖公路 518号 16号楼, Shanghai, China
\end{CJK}
}



% ------------------------------START OF DOCUMENT------------------------------

% You shouldn't need to edit code above this line for the template to work as 
% intended. Just fill in your content and it should work Just Like Magic (TM).

\begin{document}

    
    \startindent
	\makewtrtitle
	\stopindent
	
	
	\begin{singlespace}
        \begin{flushright}
        James Si Teng Liu\\
    	251 Cochrane Terrace\\
	    Milton, ON, Canada, L9T 8C8\\
	    \end{flushright}


		% TODO: CHECK ME IF IT'S STILL FIEGUTH
		Dr. Paul Fieguth, Professor and Department Chair
	
		Systems Design Engineering
	
	    University of Waterloo
	
	    Waterloo, ON
	
	    N2L 3G1\\
	
		Dear Professor Fieguth,
	\end{singlespace}
    
        I have prepared this report, ``\reportTitle" as my 3B Work Report for the Engineering team at \company. This report is the final of three that I must submit as part of my degree requirements, and it has not received any previous academic credit. This report was entirely written by me and has not received any previous academic credit at this or any other institution.\\
    	
    	FireBright designs and manufactures battery charging solutions for electric vehicles. Under the guidance of my manager Chen Jie, I was able to succcessfully produce the electrical specifications and final design of a circuit protection board.\\
    	
    	The purpose of this report is to detail the process and design choices made in the creation of this circuit protection module, as well as major points of learning through the process. \\


	    \noindent
	    Sincerely,\\
	   % \includegraphics[scale =0.2]{signature}\\
    	
    	\studentName \\
    	\studentNumber \\
        3B Systems Design Engineering

	% new page at the end of the letter.
	\newpage


	% Abstract
	%\addcontentsline{toc}{section}{Abstract}
	\begin{abstract}
    Fire safety is a critical component of battery technology, on the road to to fulfill this requirement for a new line of products, FireBright has commissioned a new module to address fire safety from both prevention and protection perspectives. This report details with the design and prototyping process of this module. The problem space is familiar territory to the company, with known sensitivity levels to tune the sensors towards. New issues include the demand for a smaller physical size, lower cost, and more robust layout.
	    
    The solution relies mainly on a architectural shift from individual \acrfull{ocp} modules for sub-assemblies to a main \acrshort{ocp} module for a single product. A smaller footprint and more design robustness is borne from making more use of digital logic in place of hardware based detection. The trade-off being made is the speed of activation of the protective measures. Given that a short-circuit protection module is also being planned, this weakness can be shored up for mission-critical circuitry in sub-assemblies, while less critical circuits still enjoy the benefit of a 'centralised' protection system.
	    
    As a result of the design and prototyping.
	\end{abstract}


	% Make a table of contents.
	\phantomsection % this adds the correct hyperref link, if you add custom ToC sections (as per \addcontentsline) USE THIS EXACT ORDERING OF COMMANDS
	\addcontentsline{toc}{section}{Table of Contents}
	\tableofcontents

	% List of figures and tables.
	\phantomsection
	\addcontentsline{toc}{section}{List of Figures}
	\listoffigures
	
	
	\phantomsection
	\addcontentsline{toc}{section}{List of Tables}
	\listoftables
	
	
	\section{Introduction}
  \pagenumbering{arabic}  % Couldn't figure out how to automate this, but leave this line in to restart page numbers at 1 and with arabic numbers.
  FireBright works in battery technology - specifically lithium-based battery solutions. Their main product is a battery hot-swap system for electric vehicles. For which FireBright designs and produces charging stations, battery packs, and vehicle specific framing. 
	
	\subsection{Problem Space}
	Due to high energy density and the flammability of the pole materials, fire safety is a critical component of lithium battery operation. For the company's new line of products, there is a need for a smaller and more robust \acrfull{cpm} that can be fitted in many different assemblies. As an Engineering Student, designing this module, prototyping, and testing the completed design was the assigned task for my work term.
	
    \paragraph{}
	The work with realizing the architecture of the module while following good engineering practice - working out details, in essence.

  \subsection{Confidentiality}
  Due to a confidentiality agreement, some details and most numbers are omitted or altered, and certain technologies used are excluded from mention.


  \paragraph{A Note on Terminology and Linguistics} - This project was done in an electrical engineering dominated environment, as such the terminology used tends toward abbreviated units and electrical shorthand. They will be expanded and explained whenever possible.
  
  \section{Understanding the Design Space}
  Following the engineering design workflow in FireBright, a number of mandates have already been made by the design team for a functional prototype to be green-lit. The mandates include the objectives of the module (the main problem it addresses), an architectural outline of major required components, and a specific microcontroller choice.

  \paragraph{}
  As a first step, the design mandates must be squared with the objectives of the module. 

  \subsection{Objectives}
  This module's main purpose is to act as a safety mechanism guarding an appliance \footnote{An assembly that uses between 100 watts and 1000 watts of power.} against failure - specifically over-current and fire. A paraphrased list of specified objectives are as follows:

  \begin{itemize}
  \item The module must not be tied to the architecture of any particular appliance in a critical way. 
  \item The module must be able to detect and report any current flow over a set limit.
  \item The module must be able to detect and report the presence of fire.
  \item Auditory and visual feedback must be provided at some level.
  \item A safety mechanism must be baked in to halt or slow thermal runaway.
  \end{itemize}

  
  \paragraph{}
  Finally, the module must be able to integrate into the existing solution system and future systems developed by FireBright.

	
	\subsection{Breakdown of Mandates\label{sec:breakdown-mandates}}
	The established scope consists of an architectural outline (including mandatory components) and two main requirement sets - electrical input/output, and board safety. Electrical I/O being both upper limits to the electrical signal strength (voltage and current) for individual inputs and outputs, and the corresponding signal profiles. Board safety would consist of absolute maximum ratings for materials, failure modes, and arc distances.
	
    \subsubsection{Architecture\label{sec:architecture}}
    As a unit, the \acrshort{cpm} was determined to be a collection of sensors feeding data into a microcontroller, which will communicate the sensor data through a standard transmission protocol used in FireBright's other products. As a system, the module will monitor the sensor data against programmed thresholds. Should the sensor readings exceed them, the module should break a relay switch controlling the monitored power circuit. The thresholds have been pre-determined, thus will not be gone into in detail. 

    TODO: architecture diagram

    \paragraph{}
    The sensors deemed as required on this module are: two temperature sensors, one smoke sensor, and an AC current sensor. Three components have been chosen ahead of time as required - a specific \acrfull{soc}, a specific RS-485 transceiver, and a specific current sensing transformer. I did not participate in these choices, thus for my part, they are treated as fixed constraints.

    \paragraph{Physical Response Systems}
    Sensory signals are also part of the requirement of the system, with controllable LED and buzzer components marked as required.

  \subsection{Input and Output} 
  There are two main types of electrical communication between system components - signal and power. The power electric behaviour is outlined as a single power supply source, followed by power modulation units that filter and deliver steady power to the components. The specifics would be determined through the design process.

  \paragraph{}
  Signal routing was derived from required component intercommunication paths \footnote{i.e. - Must have connection from sensor to SoC, SoC to transceiver, and so on.}. The final signal profiles were reached through design, testing, and analysis of each individual signal route.

  \paragraph{}
  Ultimately, input and output signal levels and limits were left as an excercise for the student to develop with reference to the specified components - both explicitly named and general (sensors needed, relay). The explicitly named components are as follows:

  \begin{itemize}
  \item \acrlong{soc} - EFM8BB1 in QFN20 package
  \item Transceiver - MAX3845 RS-485 Transceiver in SOP-8 package
  \end{itemize}
  
  \subsection{Power considerations}
  As the module is going to be a safety mechanism, it is intended to stay powered at all times, thus power draw is specifically budgeted at less than 1 watt on average.
	
  \subsection{Developing Specifications}
  A number of specifications can be extracted from the mandated components. These specifications come in the form of operating voltage level, signal profiles, and other electrical details that actualize connections between the individual components.


  \paragraph{}
  A quick connection diagram (Figure~\ref{fig:connect-diag}) helped categorize the connections. The route is transmitted from the components in the left column and is received by the components in the top row. As the original diagram was lost, a similar diagram from later in the design process (with power routing completed) is shown here. (1) denotes a 3.3V connection, (2) denotes a 5V connection, (*) denotes an analog (variable voltage level) connection. From the diagram, it was evident that most electrical connectivity is (as expected) centered around the SoC. 
  
  \begin{figure}[]
    \centering
    \caption{Connection chart detailing the input and output relationships between the components.}
    \label{fig:connect-diag}
    \includegraphics{connection-chart}
  \end{figure}
	
  Given this result, designing the sensor circuitry around the specifications of the EFM8BB1 SoC would be a fairly direct way to accomplish the objectives of the system.
  
    \subsubsection{Other Options?}
    Given the inflexible nature of the SoC being pre-selected as a fixed part of the design, forcing the SoC to behave according to the sensor output may come at significant software cost. \footnote{A cost-benefit analysis (with due consideration for the architecture) was done by the design team previous to this design cycle to select the SoC as it was likely to be the most expensive component in the module.}
  
    \paragraph{}
    To further explain, a \acrlong{soc} is a collection of Application Specific Integrated Circuits \acrshort{asic} that provide additional functionality to a CPU. For the EFM8BB1 in particular, there is an \acrfull{adc} \acrshort{asic} in the chip itself that is pre-set for a certain voltage range, but with an option to use an external voltage reference. However this requires an external voltage reference: an additional complexity to implement as a hard circuit, versus simply using the SoC as-is requiring no additional board space.

    \paragraph{}
    It turns out the operating range and behaviour of the SoC alone is sufficient to determine specifications of all sensor circuits (see section~\ref{sec:understanding-soc} for details). 

    \paragraph{What if the SoC selection is bad?}
    Is a very real and pertinent question to the overall success of the design. Given the needs of the module, the SoC selection was heavily based on unit price relative was heavily based on unit price relative was heavily based on unit price relative to the capabilities of the SoC. This will be further touched on in section~\ref{sec:select-core-comp}

	\section{Electrical Design Cycle\label{sec:devel-electr-design}} 

  Starting the design cycle with the given objectives, a simple block diagram was drawn from the connection chart to help visualize the organization of the circuits.

  TODO: block diagram of subcircuits - all subcircuits in blocks

  \paragraph{}
  Since the SoC is a single unit (with fixed electrical properties), the only way to ensure the signals between the I/O and the SoC reach agreement is to tune the I/O circuits in accord with the SoC's specifications. Furthermore, since the SoC is required to act on the sensor data, the SoC controlled devices must necessarily respond to the signals that the chip is capable of delivering. Explicitly speaking, if the chip can only provide signals at 3.3V, a switch requiring 5V cannot be directly controlled by the SoC and must be redesigned.
  
  \paragraph{}
  It follows that the limitations and capabilities of the EFM8BB chip must be fully understood in order to properly design the I/O circuitry.
  
  \subsection{Understanding the EFM8BB\label{sec:understanding-soc}}

  The EFM8BB1 is a well established microchip architecture. The datasheet is complete, and the user manual highly informative - with detailed input/output specifications and operational instructions. This is the starting point for all the sensor circuit design.


  \paragraph{}
  All conclusions in this section are drawn from the datasheet \cite{silabs:efm8bb1ds} and the reference manual\cite{silabs:efm8bb1rm} of the EFM8BB1. 

  
  \subsubsection{Pinout}
  The QFN20 package implies a 20 pin scheme. Following the pin definitions given by the datasheet, a total of 16 pins can be used for I/O purposes, with 4 reserved for power supply and programming.

  

  TODO: pinout diagram of efm8bb

  \subsection{Initial Design}

  The initial circuit diagram was based on 

	\subsection{Power}
  

  - Relay uses raw input - set input to sane value
  - speaker power is switched by the SoC, power delivered by the LDO
	
	\section{Developing Mechanical Specifications}
	
	\section{Details and Datasheets}
  T

  The operating voltage level of the EFM8BB1 family is 3.3V, thus the sensor circuitry must conform to this level. There eventually arose a need for a separate 5V signal level due to the smoke sensor's digital circuit, resolving this will be touched upon in section \ref{sec:devel-electr-design}.

  \paragraph{}
  Ultimately for each sensor, the signal strength must be scaled to the acceptable input voltage ranges of the microcontroller, and the output must be able to be controlled by the voltage signal generated by the microcontroller as well.
	\section{Understanding Limitations}
	\subsection{Resources}
  \subsubsection{Accuracy of Measurement Tools}
	\subsection{Tooling Available}
	
	\section{Component Selection}
	
	\section{Final Electrical Design}

  \section{Conclusions}

  \section{Recommendations}

  \section{Discussion}

  \subsection{Selecting Core Components\label{sec:select-core-comp}}

  Without knowing the bus 
	
	
    \printglossaries
    
    \phantomsection
    \printbibliography[heading=bibintoc]
	

\end{document}


